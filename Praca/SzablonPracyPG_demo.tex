\documentclass[nostrict]{szablonPG}

% Je�eli chcemy aby PDF mia� tytu� naszej pracy mo�emy skorzystac z tej konstrukcji
% UWAGA! zmieni� dane na w�a�ciwe!!!
\usepackage[unicode=true]{hyperref}
\newcommand\PDFtitle{Szablon Pracy PG - demo - ��������}
\newcommand\PDFauthors{��������}
\hypersetup{
  pdftitle={\PDFtitle},
  pdfauthor={\PDFauthors},
%   pdfsubject={},
%   pdfkeywords={}
 }

\usepackage{multicol} % do sk�adu w dw�ch kolumnach

%---- tylko na potrzeby tego pliku
\makeatletter
\renewcommand{\verbatim@font}{\ttfamily\small}
\makeatother
%-----

\begin{document}

\tableofcontents    % spis tre�ci
\listoffigures      % spis obrazk�w

\include{chapter1}
\chapter{Uwagi od strony technicznej}

\section{Opcje}

\begin{enumerate}
 \item \verb|strict| -- domy�lnie, klasa stara si� jak naj�ci�lej wype�nia� zalecenia
 \item \verb|nostrict| -- drobne modyfikacje typograficzne
 \begin{itemize}
  \item zmniejszenie wci�cia akapitowego z 1.25cm na 1.5em
 \end{itemize}
\end{enumerate}

\section{Wymagane pakiety}
Lista pakiet�w, kt�re s� wymagane do kompilacji (wi�kszo�� z nich jest zapewne zainstalowana
domy�lnie)
\begin{enumerate}
  \item \verb|polski| -- polonizacja \TeX'a
  \item \verb|fontenc| -- kodowanie znak�w
  \item \verb|inputenc| -- kodowanie znak�w
  \item \verb|helvet| -- wybiera font podobny do Arial
  \item \verb|geometry| -- ustawienie margines�w
  \item \verb|indentfirst| -- wci�cie pierwszego akapitu
  \item \verb|fancyhdr| -- paginacja
  \item \verb|titlesec| -- tytularia
  \item \verb|titletoc| -- formatowanie spisu tre�ci
  \item \verb|enumitem| -- wyliczenia numerowane i nienumerowane
  \item \verb|amsmath,amssymb,amsthm| -- standardowe pakiety matematyczne
  \item \verb|graphicx| -- do��czanie obrazk�w
  \item \verb|subfig| -- wiele obrazk�w na jednym rysunku
  \item \verb|caption| -- format podpisu pod obrazkiem
\end{enumerate}

\section{Fonty}

Wymaganym fontem jest Arial. Poniewa� taki font nie jest �atwo dost�pny w \LaTeX'u wi�c korzystamy
z fonta zast�pczego w pakiecie \verb|helv|. Wymagany font matematyczny nie zosta� podany. U�ywamy
zatem fontu z pakietu \verb|mathpazo|.
\begin{verbatim}
\usepackage{helvet}
\usepackage{mathpazo}
\renewcommand{\familydefault}{\sfdefault}
\end{verbatim}

Inn� wersje fontu bezszeryfowego mo�na uzyska� poprzez zrezygnowanie z pakietu \verb|helv|.

Przyk�adowy: $\sin(x)+ay^2$.
\[
 \sin(x)+ay^2
\]


%bibliografia
\bibliographystyle{plain}                       % styl bibliografii
\begin{thebibliography}{3}                      % pocz�tek �rodowiska
\addcontentsline{toc}{chapter}{Bibliografia}    % dodaje bibliografi� do spisu tre�ci
\small              % spisy i bibliografie sk�adamy mniejszym stopniem pisma
% przyk�adowy wpis
    \bibitem{Duda}      % \bibitem{etykieta}
A. Duda,\emph{Wprowadzenie do topologii}, PWN, Warszawa 1986
% nast�pna pozycja
    \bibitem{EngeSiek}
R. Engelking, K. Sieklucki, \emph{Geometria i topologia. Cz�� II. Topologia}, PWN, Warszawa 1980
% nast�pna pozycja
    \bibitem{Patk}
H. Patkowska, \emph{Wst�p do topologii}, PWN, Warszawa 1979
% nast�pna pozycja
    \bibitem{Siek}
K. Sieklucki, \emph{Geometria i topologia. Cz�� I. Geometria}, PWN, Warszawa 1979
% nast�pna pozycja
    \bibitem{Rutkowski}
Rutkowski J., \emph{Algebra Abstrakcyjna w zadaniach},  Wydawnictwo Naukowe PWN, Warszawa 2005
\end{thebibliography}                           % koniec �rodowiska
\end{document}

%EOF