%%%%%%%%%%%%%%%%%%%%%%%%%%%%POCZĄTEK CZĘŚCI KONFIGURACYJNEJ%%%%%%%%%%%%%%%%%%%%%%%%%%%%
\documentclass{beamer}
\usepackage{pgf}
\usepackage[polish]{babel}
\usepackage[utf8]{inputenc}
\usepackage[T1]{fontenc}
\usepackage{beamerthemesplit}
\usepackage{graphics,epsfig, subfigure}
\usepackage{url}
\usepackage{srcltx}
\usepackage{hyperref}
\usepackage{setspace}
\usepackage{caption}
\usepackage{helvet}

\setbeamertemplate{headline}{}


\mode<presentation>
\captionsetup{font=scriptsize,labelfont=scriptsize}
\addto\captionspolish{%
  \renewcommand{\figurename}{Rys.}%
}
\beamertemplatenavigationsymbolsempty  % Wyłączenie widoczności symboli do nawigowania rzutnikiem

%%%% Aktywacja numeracji slajdów i rysunków %%%%%%
\setbeamertemplate{footline}[frame number, footnote]
\setbeamertemplate{caption}[numbered]

%%%%%% Konfiguracja czcionki do wzorów? %%%%%%%%%
  \useinnertheme{circles}   % punktory w kształcie koła
  \usefonttheme[onlymath]{serif}
  \setbeamercovered{transparent}

%%%%%%%% Definicje nowych, pięknych kolorów PG, oraz czcionek %%%%%%%%%%
\definecolor{BackgroundPGBlue}{RGB}{1, 55, 102}
\definecolor{BluePG}{RGB}{1, 55,102}
\definecolor{RedPG}{RGB}{227,27,35}

%%%%%%%% Globalne ustawienia kolorów dla poszczególnych sekcji (tytuly, podpisy itp) oraz czcionek %%%%%%
\setbeamercolor{frametitle}{bg=white,fg=BackgroundPGBlue}
\setbeamercolor{itemize item}{fg=BluePG}
\setbeamercolor{item}{fg=BluePG}
\setbeamercolor{thead}{fg=white,bg=RedPG}
\setbeamercolor{normal text}{fg=BluePG}
\setbeamerfont{author}{size=\normalsize}
\setbeamercolor{section in toc}{fg=BluePG}
\setbeamertemplate{enumerate items}[default]
%%%%%%%% Ustawienie określonej pozycji sekcji tytułu slajdu %%%%%%%%%
\makeatletter
\setbeamertemplate{frametitle}{%
 \vspace*{0.7cm} \hspace*{3.6cm} 
  \begin{minipage}{0.65\textwidth}
    \usebeamerfont{frametitle}%
	\begin{small}
	\begin{spacing}{0.8}
	 \textbf{\insertsection}
	 \end{spacing}
	\end{small}
  \end{minipage}
  
 \vspace{0.2cm}
 
 \begin{beamercolorbox}[sep=0.3cm,wd=15cm]{thead}
 \begin{small}\textbf{\insertframetitle}
 \end{small}
 \end{beamercolorbox}
}
\makeatother


\defbeamertemplate*{title page}{customized}[1][]
{
  \begin{center}
  \vspace{1cm}
  \includegraphics[scale=0.14]{logotytul} \\ \vspace{0.6cm}
  \usebeamercolor[fg]{subtitle}
  
  \usebeamerfont{title}\textbf{\inserttitle} \\ \vspace{0.35cm}
  \usebeamerfont{author} Autor: \insertauthor \\ \vspace{0.4cm}
  \usebeamerfont{institute}\insertinstitute \\
  \usebeamerfont{date}\insertdate\par
  \end{center}
}

%%%%% Informacje do strony tytułowej %%%%%%%
\title{Tytuł Prezentacji}
\author{Autor prezentacji \\ Drugi autor/Promotor}
\institute{Nazwa wydziału \\ Politechnika Gdańska}
\date{Data}



\begin{document}

%%%%%%% Wygenerowanie strony tytułowej %%%%%%%
\setbeamercolor{background canvas}{bg=BackgroundPGBlue}%ustawienie pięknego granatowego koloru PG
%\titlegraphic{\includegraphics[scale=0.12]{logotytul}}
\setbeamertemplate{footline}{} % Deaktywacja stopki z numeracja
\frame{
\maketitle
}

%%%%%% Dołączenie logo PG w lewy górny róg prezentacji %%%%%%%
\logo{\pgfputat{\pgfxy(-10,7.5)}{\pgfbox[center,base]{\includegraphics[scale=0.12]{logo}}}}  

\setbeamertemplate{footline}{%
  \begin{flushright}
  \tiny \normalfont \insertframenumber/\inserttotalframenumber
  \end{flushright}
}
 % Ponowna aktywacja stopki z numeracja
\addtocounter{framenumber}{-1}
%%%%%% Powrót do białego tła %%%%%%%%
\setbeamercolor{background canvas}{bg=white}

\setbeamertemplate{section in toc}{\inserttocsectionnumber.~\inserttocsection}

\setbeamerfont{section in toc}{shape=\bfseries}

%%%%%%%%%%%%%%%%%%%%%%%%KONIEC CZĘŚCI KONFIGURACYJNEJ%%%%%%%%%%%%%%%%%%%%%%%%%%%%%%%%%%


%%%% Początek -  Nasza prezentacja  %%%%%%

\begin{frame}
\tableofcontents  % Wygenerowanie spisu treści (na podstawie sekcji i podsekcji)
\frametitle{Plan prezentacji}
\end{frame}


%% Przykładowe sekcje prezentacji %%

\section{Pierwsza sekcja prezentacji}
\subsection{Pierwsza podsekcja}
\subsection{Druga podsekcja}

\section{Druga sekcja prezentacji}
\subsection{Podsekcja}


\section{Trzecia sekcja prezentacji}
\subsection{Podsekcja}


\section*{Literatura}
\frame{
\small %% lokalne zmniejszenie rozmiaru czcionki
\frametitle{Materiały źródłowe}
\begin{enumerate}
\item Autor J., Autor D.: Tytuł publikacji, Wydawnictwo, Miasto i Data
\item Autor J., Autor D.: Tytuł publikacji, Wydawnictwo, Miasto i Data
\item Autor J., Autor D.: Tytuł publikacji, Wydawnictwo, Miasto i Data
\item Autor J., Autor D.: Tytuł publikacji, Wydawnictwo, Miasto i Data
\item Autor J., Autor D.: Tytuł publikacji, Wydawnictwo, Miasto i Data
\item Autor J., Autor D.: Tytuł publikacji, Wydawnictwo, Miasto i Data
\item Autor J., Autor D.: Tytuł publikacji, Wydawnictwo, Miasto i Data
\end{enumerate}
}



%%%%%%% Ostatni slajd - motto PG %%%%%%%%
\begin{frame}[plain]
\addtocounter{framenumber}{-1}
\vspace{1cm}
\begin{center}
\includegraphics[keepaspectratio=true,scale=0.25]{logokoniec.png}
\end{center}
\end{frame}

\end{document}
